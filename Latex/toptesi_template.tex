\documentclass[%
	corpo=11pt,
    twoside,
    stile=classica,
    oldstyle,
    tipotesi=custom,
    greek,
    evenboxes,
]{toptesi}
%%%%%%%%%%%%%%%%%%%%%%%%%%%%%%%%%%%%%%%%%%%%%%%%%%%%

\usepackage{listings}
\usepackage{xcolor}

\definecolor{codegreen}{rgb}{0,0.45,0}
\definecolor{codegray}{rgb}{0.5,0.5,0.5}
\definecolor{codepurple}{rgb}{0.58,0,0.82}
\definecolor{backcolour}{rgb}{0.95,0.95,0.95.22}

\lstdefinestyle{mystyle}{
    backgroundcolor=\color{backcolour},   
    commentstyle=\color{codegreen},
    keywordstyle=\color{magenta},
    numberstyle=\tiny\color{codegray},
    stringstyle=\color{codepurple},
    basicstyle=\ttfamily\footnotesize,
    breakatwhitespace=false,         
    breaklines=true,                 
    captionpos=b,                    
    keepspaces=true,                 
    numbers=left,                    
    numbersep=5pt,                  
    showspaces=false,                
    showstringspaces=false,
    showtabs=false,                  
    tabsize=2
}

\lstset{style=mystyle}



\usepackage[utf8]{inputenc}
\usepackage[T1]{fontenc}
\usepackage{lmodern}

\usepackage{hyperref}
\hypersetup{%
    pdfpagemode={UseOutlines},
    bookmarksopen,
    pdfstartview={FitH},
    colorlinks,
    linkcolor={blue},
    citecolor={blue},
    urlcolor={blue}
  }

%%%%%%% use PDFLATEX 

\usepackage{stix}
\usepackage{epigraph}

\usepackage{lipsum} %to insert random text

\usepackage{geometry} %for the margins
\newcommand\fillin[1][4cm]{\makebox[#1]{\dotfill}} %for the dotted line in the frontispiace

\usepackage{dcolumn}
\newcolumntype{d}{D{.}{.}{-1} } %to vetical align numbers in tables, along the decimal dot

\usepackage{amsmath}

\usepackage{natbib} % for the bibliography
\bibliographystyle{plainnat}




%%%%%%%%%%%%%%%%%%%%%%%%%%%%%%%%%%%%%%%%%%%%%%%%
%%%%%%%%%%%%%%%%%%%%%%%%%%%%%%%%%%%%%%%%%%%%%%%%



\begin{document}\errorcontextlines=9
%\english

\input{./title.tex} %the frontispiece

%%%%%%% Dedication
\ifclassica%
\begin{dedica}

    $\heartsuit$\ All'autrice di "Passi" %Alla mia regina
\end{dedica}
%%%%%%% 

\sommario%summary
%Here goes the abstrat of your thesis

Le immagini sono sempre state fondamentali per l'uomo, che da sempre le utilizza per ricordare, illustrare, comunicare, etc. Per questo motivo, sfruttarle nel migliore dei modi è sempre stato un problema di grande interesse. Esistono dei metodi di elaborazione digitali detti "di filtraggio" il cui scopo è modificare l'immagine, estraendone alcuni elementi, nascondendoli o facendoli risaltare così da migliorarla, cioè renderla quanto più utile agli scopi richiesti.\\
Durante questa trattazione, dopo una breve intoduzione che servirà a dare una visione d'insieme del campo nel quale ci si sta introducendo, sarà illustrato il concetto di immagine digitale, di filtro ed in particolare di equazioni alle derivate parziali, di come esse giochino un ruolo fondamentale nella scrittura dei filtri e la loro implementazione al calcolatore.\\
Nello specifico, nella parte introduttiva si parlerà di storia delle immagini digitali e del filtraggio analogico e digitale, dei principali problemi affrontati e di alcune delle più semplici elaborazioni digitali. Verranno fornite alcune nozioni introduttive, come il concetto di immagine come funzione matematica e quindi di immagine digitale, capire come è codificata, si parlerà quindi di pixel, del concetto di risoluzione e di quanto essa influisca sullo spazio di archiviazione. In fine saranno introdotti i filtri, intesi come convoluzioni tra la funzione immagine e la funzione filtro.\\
Inizierà poi la trattazione più specifica, che sarà il fulcro di questo studio, ossia l'implementazione del metodo Perona-Malik. Per fare ciò verrà illustrato il metodo delle differenze finite per l'approssimazione delle derivate con relativa analisi dell'errore di troncamento. Questo risultato sarà sfruttato per l'implementazione dell'equazione del calore tramite uno script MATLAB che sarà poi modificato per implementare il metodo Perona-Malik.   



%%%%%%%%%%%%%%%%%%%%%%%%%%%%%%%%%%%%%%%%%%%%%%%%
%%%%%%%%%%%%%%%%%%%%%%%%%%%%%%%%%%%%%%%%%%%%%%%%

\ringraziamenti%acknowledgements
%Acknowledge the people you love and/or work with
I candidati ringraziano vivamente il Granduca di Toscana per i mezzi messi loro a disposizione, ed il signor Von Braun, assistente del prof.~Albert Einstein, per le informazioni riservate che egli ha gentilmente fornito loro, e per le utili discussioni che hanno permesso ai candidati di evitare di riscoprire l'acqua calda.

Questa parte è rimasta inalterata da com'era nel template.

%%%%%%%%%%%%%%%%%%%%%%%%%%%%%%%%%%%%%%%%%%%%%%%%
%%%%%%%%%%%%%%%%%%%%%%%%%%%%%%%%%%%%%%%%%%%%%%%%

%\tablespagetrue
\figurespagetrue%to include the list of tables
%and the list of figures - yuo can comment these commands

\indici%table of content
%It automatically generated

%%%%%%%%%%%%%%%%%%%%%%%%%%%%%%%%%%%%%%%%%%%%%%%%
%%%%%%%%%%%%%%%%%%%%%%%%%%%%%%%%%%%%%%%%%%%%%%%%

%Citation
%If you feel like a poetic guy!
\ifclassica   
\begin{citazioni}
    \textit{If you cannot understand my\\argument, and declare}\\
    it's Greek to me\\
    \textit{you are quoting Shakespeare.}
    
    [\textsc{B. Levin}, Quoting Shakespeare]\vspace{1em}
\end{citazioni}
\fi

%%%%%%%%%%%%%%%%%%%%%%%%%%%%%%%%%%%%%%%%%%%%%%%%
%%%%%%%%%%%%%%%%%%%%%%%%%%%%%%%%%%%%%%%%%%%%%%%%

\mainmatter

\part{Introduzione}
\section{Storia del filtraggio e principali problemi affrontati}

L'essere umano, da sempre, sfrutta i propri sensi per relazionarsi con il mondo.\\
Il suo primo riferimento in particolare è la vista, senza la quale egli si sente perso, smarrito; per questo è fondamentale sviluppare un linguaggio  visivo che sia efficace.\\
La società contemporanea sfrutta immagini di vario genere (fotografie, radiografie, ecografie, poster pubblicitari, progetti, etc.) per comunicare messaggi e/o per rendere chiari progetti, idee, situazioni.\\
\`E quindi un problema sempre di grande interesse cercare di sfruttarle al meglio. A tal proposito esistono alcuni metodi di \textbf{filtraggio}, la cui finalità è quella di migliorare la qualità delle immagini, metterne in risalto determinate caratteristiche o nasconderne delle altre.\\


\subsection{Filtraggio analogico}
Anche prima dell'avvento dei computer esisteva l'elaborazione delle immagini. Esistevano infatti dei procedimenti, detti \textit{"mascherature"}, che servivano a ridurre o esaltare le differenze di luce nella foto.\\
La mascheratura avveniva durante la fase di stampa su carta, ossia quando il negativo veniva proiettato sulla carta fotografica mediante l’ingranditore. 
Qui, l'operatore utilizzava una serie di metodi per far sì che su certe zone della carta andasse più o meno luce di quella che sarebbe arrivata passando attraverso il negativo.
Ad esempio la tecnica di mascheratura per la correzione del contrasto prevedeva la sovrapposizione a registro della stessa immagine sia in positivo che in negativo, con lo scopo di ridurre o accentuare il contrasto.
\newpage

\begin{figure}[htb] \centering
\includegraphics[scale=4]{Pictures/analogico_originale.jpg}
\qquad\qquad
\includegraphics[scale=4]{Pictures/analogico_filtrata.jpg}
\caption{Mascheratura per la correzione del contrasto. \cite{fodde}}
\end{figure}

\noindent
Partendo da questa si sono sviluppate tecniche specifiche come la mascheratura detta ad \textit{"altissimo contrasto"} permetteva di avere foto con soli bianchi e soli neri, molto simile a quello che oggi chiamiamo \textit{"posterizzazione"}\\ 
Il viraggio invece serviva a dare un tono di colore alla foto, che restava comunque
un bianco e nero: famoso il viraggio tono seppia.




\subsection{Immagini digitali}
Le immagini digitali trovarono le prime applicazioni sui giornali negli anni 20 \footnote{\cite{storia}}. Non esistevano veri e propri computer, il segnale era trasmesso tramite telegrafo simulando dei mezzitoni.
In particolare, il sistema di trasmissione di immagini via cavo Bartlane era una tecnica inventata nel 1920 per trasmettere immagini di giornali digitalizzate su linee sottomarine tra Londra e New York.\\
Il sistema di trasmissione di immagini via cavo Bartlane generava sia sul trasmettitore che sul ricevitore una scheda dati perforata o un nastro che veniva ricreato come immagine.\\
\begin{figure}[htb] \centering
\includegraphics[scale=0.6, trim = 0 1.1cm 0 0, clip]{Pictures/nastro Bartlane.jpg}
\caption{Nastro dati codificante un'immagine.}\label{fig:figura}
\end{figure}

\noindent
In questo modo riusciva a codificare immagini in bianco e nero in cinque diverse gradazioni di grigio, capacità poi ampliata a 15 gradazioni nel 1929.

\begin{figure}[htb] \centering
\includegraphics[scale=0.5, trim = 0 1.1cm 0 0, clip]{Pictures/img del 1921 a 5 gradazioni di grigio.jpg}
\qquad\qquad
\includegraphics[scale=1.7, trim = 0 1.1cm 0 0, clip]{Pictures/img del 1929 a 15 gradazioni di grigio.jpg}
\caption{La prima immagine risale al 1921 ed è composta da 5 gradazioni di grigio.\\ 
La seconda è del 1929 ed è composta da 15 gradazioni di grigio. \cite{storia} }\label{fig:figura}
\end{figure}
\noindent
Questa, in qualche modo, è la nascita delle immagini digitali, anche se non sono trattate da computer e quindi codificate in maniera differente. Inoltre in questa fase non abbiamo una vera e propria elaborazione delle immagini, ma solo una trasmissione e la successiva stampa.

\vspace{1em} \noindent
Nel 1957, Russell A. Kirsch produsse un dispositivo che generava dati digitali che potevano essere archiviati in un computer; questo utilizzava uno scanner a tamburo e un tubo fotomoltiplicatore. Negli anni immediatamente successivi, tale invenzione portò a notevoli sviluppi.


\subsection{Filtraggio digitale}
Negli anni 60 del XX secolo vari laboratori come il Jet Propulsion Laboratory, il Massachusetts Institute of Technology, i Bell Laboratories e l'Università del Maryland svilupparono molte tecniche di filtraggio digitale di immagini (o trasformazione di immagini digitali come spesso veniva chiamata) al fine di evitare le debolezze operative delle fotocamere a pellicola per missioni scientifiche e militari\footnote{\cite{storia}}. Queste trovarono poi applicazione in immagini satellitari, immagini medicali, videocitofono, riconoscimento ottico dei caratteri, e miglioramenti fotografici.\\
\noindent
Il costo dell'elaborazione in quel periodo era piuttosto alto per via del prezzo elevato delle apparecchiature utilizzate. Le cose cambiarono negli anni settanta, quando sul mercato furono resi disponibili computer più economici e hardware dedicato.\\
Le immagini allora potevano essere elaborate in tempo reale, l'elaborazione digitale sostituì così i vecchi metodi a pellicola per molti scopi.\\

\noindent
Negli anni 2000, grazie all'avvento di computer più veloci, il filtraggio digitale divenne la forma più comune di elaborazione delle immagini e, in generale, divenne il metodo più utilizzato data la sua versatilità e il basso costo.
\noindent
Le tecnologie utilizzate per il filtraggio delle immagini digitali per applicazioni mediche è stata inserita nel 1994 nella Hall of Fame della Space Foundation, esse infatti permisero di sviluppare le tecnica della tomografia computerizzata assiale, meglio nota come TAC molto utilizzata anche ai giorni nostri in radiologia.\\

\chapter{Nozioni introduttive}
%Viviamo in un era digitale, l
La società moderna è una società digitale. Le immagini passano generalmente per un calcolatore prima di essere stampate, o in ogni caso possono essere sempre scannerizzate (con strumenti più o meno precisi) così da averne una copia digitale.
\`E in questa fase che l'immagine subisce il processo di filtraggio, nel calcolatore, quando è in formato digitale, Per capire cos'è un filtro occorre dunque chiedersi cosa sia un'immagine digitale.

\section{Immagini digitali}
%Per capire come codificare un'immagine e memorizzarla in formato digitale ci chiediamo prima che cos'è un'immagine.
\`E necessario comprendere a fondo cosa sia un oggetto, per poterlo poi implementare in un calcolatore, a tal scopo è utile quindi capire esattamente che cos'è un'immagine

\begin{quote}
\epigraph{\textit{Forma esteriore degli oggetti corporei, in quanto viene percepita attraverso il senso della vista, o si riflette – come realmente è, o variamente alterata – in uno specchio, nell’acqua e sim., o rimane impressa in una lastra o pellicola o carta fotografica.}}{Vocabolario Treccani}
\end{quote}

%\begin{citazioni}
%    \textit{Forma esteriore degli oggetti corporei, in quanto viene percepita attraverso il senso della vista, o si riflette – come realmente è, o variamente alterata – in uno specchio, nell’acqua e sim., o rimane impressa in una lastra o pellicola o carta fotografica.}
    
%    [\textsc{Vocabolario Treccani}, Definizione di immagine]\vspace{1em}
%\end{citazioni}


\noindent
Un'immagine viene rappresentata, impressa quindi su superfici, cioè oggetti bidimensionali, di dimensioni finite e le vediamo perchè i nostri occhi percepiscono il susseguirsi di colori diversi. Come codificare tali entità?
Come per tutti gli oggetti reali, sebbene abbiano dimensioni finite, le immagini sono distruibuzioni continue, o per meglio dire, il susseguirsi delle gradazioni di colore avviene in una maniera che possiamo considerare come continua. Questo è il primo problema che ci si pone quando si pensa a come codificare delle immagini.\\
La soluzione più largamente utilizzata è anche quella più semplice ed intuitiva, ossia di discretizzare tale distribuzione di colori. Si divide l'immagine con una griglia e ad ogni casella, che d'ora in poi chiameremo \textbf{"pixel"}, assegnamo un colore.
\`E ovvio che così facendo si perdono dei dettagli, la quantità di dettagli che riusciamo a conservare può variare enormemente, una minima quantità di dettagli si perde sempre ma è un prezzo che vale la pena pagare.
\newpage
Facciamo un esempio:

%\ref{fig:figuraa}. 
\begin{figure}[htb] \centering
\includegraphics[scale=0.03]{Pictures/in ricordo del pinguino cameriere.png}
%\caption{Fiamme.}\label{fig:figura}
\qquad\qquad
\includegraphics[scale=0.03]{Pictures/canvas8x8.png}
\caption{Confronto tra immagine originale e immagine codificata utilizzando una griglia 4x4.}\label{fig:figura}
\end{figure}

%\ref{fig:figura}. 
\begin{figure}[htb] \centering
\includegraphics[scale=0.03]{Pictures/in ricordo del pinguino cameriere.png}
\qquad\qquad
\includegraphics[scale=0.03]{Pictures/canvas80x80.png}
\caption{Confronto tra immagine originale e immagine codificata utilizzando una griglia 80x80.}\label{fig:figura}
\end{figure}

\noindent
\`E semplice vedere come ad una griglia più fitta corrisponda una miglior qualità dell'immagine, questo è il concetto di \textbf{"risoluzione di un'immagine"}. 
Una miglior risoluzione però costa, come anticipato, in termini di memoria. Una griglia 4x4 corrisponde a 16 pixel, ad una griglia 80x80 corrispondono invece 6400 pixel! Questo vuol dire che la seconda immagine pesa 400 volte di più della prima.

\vspace{1em} \noindent
Volendo definire in maniera più precisa che cosa è un'immagine digitale, diremmo che quest'ultima è una funzione da $\mathbb R^2$ in $\mathbb R^3$ cioè, date in input due coordinate, essa restituisce un colore (che è formato da 3 canali RGB). Se però l'immagine è in bianco e nero la questione si semplifica: l'immagine diventa una funzione da $\mathbb R^2$ in $\mathbb R$, dal momento che per codificare un colore appartenente alla scala di grigio basta un solo canale: il livello di luminosità. Difatti, la riduzione da tre ad un solo canale rappresenta un grosso vantaggio, permettendo di diminuire di due terzi lo spazio di memoria occupato.

\section{Cosa sono i filtri}
Una volta capito che un'immagine è una funzione possiamo definire un filtro come una seconda funzione che convoluta alla prima da il risultato richiesto.\\
Una equazione alle derivate parziali (PDE) esprime una evoluzione. Sia u la nostra immagine e $u_0$ lo stato in cui si trova inzialmente, allora per convoluzione possiamo dire che 

\begin{equation} \label{eq:eq3}
u(x)=\frac{1}{w(x)}\int\int d(x-\xi)\Tilde{d}(u_0(x) -u_0(\xi))u_0(\xi)d\xi.
\end{equation}

\centering con  $w(x) = \int\int d(x-\xi)\Tilde{d}(u_0(x) -u_0(\xi))d\xi$\newline

\raggedright

In matematica, la \textbf{convoluzione} è un'operazione tra due funzioni di una variabile che consiste nell'integrare il prodotto tra la prima e la seconda traslata di un certo valore.

E questo è a tutti gli effetti un filtro. Il problema adesso è far eseguire questi calcoli ad un calcolatore, il quale non è in grado di lavorare con oggetti continui e richiede quindi di alcune approssimazioni per "discretizzare" il problema.

A tal proposito la definizione di convoluzione può facilmente essere discretizzata parlando di successioni anzichè di funzioni ed operando una sommatoria invece di un integrale.

$$
(f*g)[n]\ {\stackrel {{\mathrm {def}}}{=}}\ \sum _{{m=-\infty }}^{\infty }f[m]\,g[n-m]=\sum _{{m=-\infty }}^{\infty }f[n-m]\,g[m].
$$

Altri metodi ed approssimazioni saranno poi approfonditi durante la trattazione del problema

%Definiti questi concetti siamo pronti ad iniziare la trattazione vera e propria


\chapter{Principali problemi di elaborazione digitale}

%\section{Principali problemi di elaborazione digitale}

Al giorno d'oggi i computer permettono una vasta gamma di elaborazioni digitali, ce ne sono alcuni, più semplici ma di notevole interesse, ormai già perfezionati ed altri che sono ancora oggetto di studio.

\section{Rotazioni, riflessioni,etc}
Le più semplici in assoluto riguardano "movimenti rigidi" come la traslazione o la rotazione, o anche omeomorfismi come la riflessione. Questi sono molto utili per introdurre i primi concetti matematici, come l'impiego di matrici.\\
Definiamo una matrice di trasformazione che moltiplicata per un vettore di coordinate ci restituisca un altro vettore di coordinate. Questo vuol dire che quel punto va spostato dalle coordinate in cui si trovava a quelle appena calcolate. Ad esempio:

$$
{\begin{bmatrix}
-1&0&0\\
0&1&0\\
0&0&1
\end{bmatrix}}
$$

\noindent
\`E una matrice di riflessione lungo l'asse verticale, infatti: 

$$
{\begin{bmatrix}
-1&0&0\\
0&1&0\\
0&0&1
\end{bmatrix}}
\begin{bmatrix}
x\\y\\1
\end{bmatrix}
=
\begin{bmatrix}
-x&y&1
\end{bmatrix}
$$	

\vspace{1em} \noindent
Cioè ogni punto rimane alla stessa quota ma cambia la propria x con -x, che è esattamente ciò che intendiamo per riflessione lungo l'asse verticale.

\newpage \noindent 
Altri esempi possono essere:

\begin{align*}
&\begin{bmatrix}
1&0&0\\
0&-1&0\\
0&0&1
\end{bmatrix}&~&
\text{Riflessione lungo l'asse orizzontale}\\
\\
&\begin{bmatrix}
2&0&0\\
0&1.5&0\\
0&0&1
\end{bmatrix}&~&
\text{Scalamento}\\
\\
&\begin{bmatrix}
\cos(\theta )&\sin(\theta )&0\\
-\sin(\theta )&\cos(\theta )&0\\
0&0&1
\end{bmatrix}&~&
\text{Rotazione di un angolo theta}\\
\end{align*}



\section{Cambio prospettiva}
Una delle trasformazioni più semplici è quella del cambio prospettiva. Una volta compresa questa trasformazione si è fatto anche un primo passo per parlare di warping, di cui parleremo più avanti.

\vspace{1em} \noindent
Supponiamo di avere la foto di un documento poggiato su una scrivania e di volerla migliorare in modo da estrarne solo il documento e che i suoi bordi concidano quindi con i bordi dell'immagine.

\vspace{1em} \noindent
In genere la cosa più semplice ed affidabile è quella di far scegliere i 4 angoli del documento ad un utente. Volendo automatizzare il processo possiamo però scegliere un'altra strada: ossia estrarre i bordi dell'immagine e cercare tra questi quelli che formano un trapezio, presumibilmente quello sarà il documento.

\vspace{1em} \noindent
Ottenuti i 4 angoli, calcoliamo la lunghezza e la larghezza della nostra nuova immagine. Per fare ciò iniziamo con il calcolare la lunghezza dei 4 lati, questi li otterremo banalmente con il teorema di Pitagora. A questo punto prendiamo i lati a due a due non adiacenti, cioè che non hanno vertici in comune e ne confrontiamo le lunghezze.\\
Per ogni coppia prendiamo la lunghezza maggiore (si può anche usare quella minore ma non dilunghiamoci su questa scelta).

\vspace{1em} \noindent
Ora siamo in grado di calcolare la matrice di trasformazione. Essa sarà del tipo:

\begin{align*}
&\begin{bmatrix}
a_1&a_2&b_1\\
a_3&a_4&b_2\\
c_1&c_2&1
\end{bmatrix}
\text{ dove }
\begin{bmatrix}
a_1&a_2\\
a_3&a_4
\end{bmatrix}
\text{ definisce rotazione e scalamento, } 
\begin{bmatrix}
b_1\\
b_2
\end{bmatrix}
\text{ è un vettore di}\\
&\text{traslazione e} 
\newline
\begin{bmatrix}
c_1&c_2
\end{bmatrix}
\text{ è un vettore di proiezione (che è nullo se il riquadro iniziale e quello}\\
&\text{finale coincidono).}\\
\end{align*}

\vspace{-3em}

\begin{align*}
&\text{L'intento è che moltiplicando tale matrice per il vettore  }
\begin{bmatrix}
\text{coordinata x}\\
\text{coordinata y}\\
1
\end{bmatrix}
\text{ otterremo il vettore}\\
&\begin{bmatrix}
\text{nuova coordinata x*k}\\
\text{nuova coordinata y*k}\\
k
\end{bmatrix}
\text{ eseguendo questa operazione per tutti i punti, costruiamo l'immagine}\\
&\text{desiderata.}
\end{align*}

\section{Morphing}
Il morphing è uno dei primi effetti digitali sviluppati dall'industria cinematografica e consiste nella trasformazione fluida, graduale e senza soluzione di continuità tra due immagini di forma diversa, che possono essere oggetti, persone, volti, paesaggi.

\vspace{1em} \noindent
Il morphing non è altro che l'uso in contemporanea di una dissolvenza incrociata e di un effetto di deformazione chiamato warping (termine inglese che significa appunto deformazione).

\vspace{1em} \noindent
Per operare il warping si definiscono sull'immagine di partenza dei "punti chiave" che possono essere uniti tra di loro con delle linee e si definiscono sull'immagine di destinazione i corrispondenti punti e di conseguenza le corrispondenti linee.\\ 
\begin{figure}[htb] \centering
\includegraphics[scale=0.5, trim = 0 1.1cm 0 0, clip]{Pictures/Striscia_morphing.jpg}
\caption{Processo di morphing con alcuni risultati intermedi.}\label{fig:figura}
\end{figure}

\noindent
Durante la dissolvenza dall'immagine iniziale a quella finale, le immagini vengono deformate facendo in modo che ciascun punto chiave si muova lungo il percorso che porta dalla sua posizione nell'immagine di partenza alla posizione del corrispondente punto nell'immagine di arrivo.\\
\part{Trattazione del problema}
\section{Equazione del calore e la sua approssimazione numerica}
L'equazione del calore, come facilmente intuibile, è stata formulata per determinare l'evoluzione di un sistema isolato che presenta al suo interno una data distribuzione di calore. La sua rappresentazione differenziale è la seguente:\\
$$
\begin{cases}
\frac{\partial u}{\partial t}(t,x)-\Delta u(t,x) = 0 \ x \in \mathbb R^2, t\ge 0 \ .\\ 
u(0,x) = u_0(x)\ . \\
\end{cases}
$$
Euristicamente, non è difficile pensare che si possano codificare (pensando ad un'immagine in bianco e nero) i pixel più luminosi come punti \textit{"più caldi"}, mentre quelli più scuri come punti \textit{"più freddi"} ed applicare così l'equazione del calore all'immagine.\\
Mediante uno script MATLAB si può dunque osservare come quest'ultima operi nella pratica, sfruttando una approssimazione alle \textbf{differenze finite } utile per il calcolo delle derivate.\\
\vspace{-0.5em}
\subsection{Metodo delle differenze finite}
\raggedright
Il metodo delle differenze finite, come anticipato, viene impiegato nel calcolo approssimato delle derivate. 
\vspace{-0.5em}
\begin{definizione}

Data una generica funzione u, consideriamone lo sviluppo di Taylor\\
\vspace{0.25em}
\centering
$u(x_i+\Delta(x))=u(x_i)+u'(x_i)\Delta(x)+\frac{1}{2}u''(x_i)\Delta(x)^2+o(h^2)$ \\
\vspace{0.25em}
\raggedright
La scelta adottata per la suddetta approssimazione risulterà dunque essere:\\
\vspace{0.25em}
\centering 
$\frac{du}{dx}\approx\frac{u(x+\Delta(x))-u(x)}{\Delta(x)} \approx \frac{u_{i+1} - u_i}{\Delta(x)} $.
\footnote{\cite{FD}}\\
\vspace{0.25em}
\raggedright
Tale approssimazione è detta \textbf{differenza finita in avanti}. Analogamente si trova, come approssimazione altrettanto valida la \textbf{differenza finita all'indietro}:\\
\vspace{0.25em}
\centering 
$\frac{du}{dx}\approx\frac{u(x)-u(x-\Delta(x))}{\Delta(x)} \approx \frac{u_i - u_{i-1}}{\Delta(x)} $.\\
\vspace{0.25em}
\raggedright
\end{definizione}
\begin{definizione}
Consideriamo adesso gli sviluppi di Taylor che hanno portato alle formule di approssimazione alle differenze finite appena viste e sottraiamo membro a membro.\\
\vspace{0.25em}
\centering
$u(x_i+\Delta(x))=u(x_i)+u'(x_i)\Delta(x)+\frac{1}{2}u''(x_i)\Delta(x)^2+\frac{1}{6}u'''(x_i)\Delta(x)^3+o(\Delta(x)^3)$ \\
\vspace{0.25em}
$u(x_i-\Delta(x))=u(x_i)-u'(x_i)\Delta(x)+\frac{1}{2}u''(x_i)\Delta(x)^2-\frac{1}{6}u'''(x_i)\Delta(x)^3+o(\Delta(x)^3)$ \\
$\Downarrow$\\
$u(x_i+\Delta(x))-u(x_i-\Delta(x))=+2u'(x_i)\Delta(x)+2\frac{1}{6}u'''(x_i)\Delta(x)^3+o(\Delta(x)^3)$ \\
\vspace{0.25em}
\raggedright
Questi conti inducono l'approssimazione:\\
\vspace{0.25em}
\centering 
$\frac{du}{dx}\approx\frac{u(x+\Delta(x))-u(x-\Delta(x))}{2\Delta(x)} \approx \frac{u_{i+1} - u_{i-1}}{2\Delta(x)}$\\
\vspace{0.25em}
\raggedright
Tale approssimazione è detta \textbf{differenza finita centrata}.
\end{definizione}

\vspace{1em}
Dal momento che la derivata seconda coincide con la derivata della derivata prima, è intuitivo dare la seguente definizione per la sua approssimazione numerica:
\begin{definizione}
\noindent
Consideriamo la funzione $u$ e l'approssimazione della sua derivata alle differenze finite $\frac{du}{dx}=\frac{u_{i+1} - u_i}{\Delta(x)}$, applicando ad esso il metodo delle differenze finite troviamo: 
\begin{align*}
\frac{d^2u}{dx^2} \approx &
\frac{
(\frac{u_{i+1} - u_i}{\Delta(x)})_{i+1} 
- 
(\frac{u_{i+1} - u_i}{\Delta(x)})_i
}{\Delta(x)}
=
\frac{
\frac{(u_{i+1} - u_i)_{i+1}}{\Delta(x)} 
-   
\frac{(u_{i+1} - u_i)_i}{\Delta(x)}
}{\Delta(x)}
\\
\vline&
\\
= &
\frac{
\frac{(u_{i+2} - u_{i+1})}{\Delta(x)} 
- 
\frac{(u_{i+1} - u_i)}{\Delta(x)}
}{\Delta(x)}
=
\frac{
\frac{(u_{i+2} - u_{i+1}) 
- 
(u_{i+1} - u_i)}{\Delta(x)}
}{\Delta(x)}
\\
\vline
\\
= &
\frac{
(u_{i+2} - u_{i+1}) 
- 
(u_{i+1} - u_i)}
{\Delta(x)^2}
=
\frac{
 u_{i+2} - u_{i+1} 
-u_{i+1} + {u_i}}
{\Delta(x)^2}
.\\
\end{align*}
\raggedright
\end{definizione}
\begin{osservazione}
Per ottenere una stima accurata è bene utilizzare un valore di $\Delta(x)$ quanto più basso possibile. La migliore è guardare la differenza tra un pixel e quello adiacente ossia $\Delta(x)=1$, ma allora 

$$\frac{d^2u}{dx^2} \approx
\frac{
 u_{i+2} - u_{i+1} 
-u_{i+1} + u_i}
{\Delta(x)^2} = u_{i+2} -2 u_{i+1} + u_i.$$

\`E chiaro che scorrendo tutti gli indici questo è equivalente a $u_{i+1} -2 u_i + u_{i-1}.$
\end{osservazione}

In sintesi si può utilizzare per il calcolo delle derivate seconde che compongono il laplaciano l'approssimazione\\
$$\frac{d^2u}{dx^2} \approx u_{i+1} -2 u_i + u_{i-1}$$
\newpage
\subsubsection{Errore di troncamento}

Vale la pena di fare qualche considerazione sull'errore di troncamento che si commette adottando queste approssimazioni.\\

\begin{definizione}
Definiamo \textbf{errore} o più precisamente \textbf{errore assoluto} $\epsilon$ di un'approssimazione $u'$ di un valore d'interesse $u$ la differenza $\epsilon=u-u'$, cioè tra il valore esatto e quello approssimato.
\end{definizione}
Per quanto riguarda le derivate prime basta guardare agli sviluppi di Taylor che ci hanno portato alle loro approssimazioni per osservare che\\
\begin{osservazione}
Applicando la definizione di errore assoluto alle diverse approssimazioni finora definite per le derivate di primo grado troviamo:
\begin{itemize}
    \item Differenza finita in avanti\\
    \centering
    $u(x_i+\Delta(x))=u(x_i)+u'(x_i)\Delta(x)+\frac{1}{2}u''(x_i)\Delta(x)^2+o(\Delta(x)^2)$ \\ $\Downarrow$\\
    $\epsilon=\frac{u(x_i+\Delta(x))-u(x_i)}{\Delta(x)} -u'(x_i)\approx \frac{1}{2}u''(x_i)\Delta(x)$\\
    \raggedright
    \item Differenza finita all'indietro\\
    \centering
    $u(x_i-\Delta(x))=u(x_i)-u'(x_i)\Delta(x)+\frac{1}{2}u''(x_i)\Delta(x)^2+o(\Delta(x)^2)$ \\ $\Downarrow$\\
    $\epsilon=\frac{u(x_i)-u(x_i-\Delta(x))}{\Delta(x)} -u'(x_i)\approx \frac{1}{2}u''(x_i)\Delta(x)$\\
    \raggedright
    \item Differenza finita centrata\\
    \centering
    $u(x_i+\Delta(x))-u(x_i-\Delta(x))=+2u'(x_i)\Delta(x)+2\frac{1}{6}u'''(x_i)\Delta(x)^3+o(\Delta(x)^3)$ \\ $\Downarrow$\\
    $\epsilon=\frac{u(x_i+\Delta(x))-u(x_i-\Delta(x))}{2\Delta(x)} -u'(x_i)\approx +\frac{1}{6}u'''(x_i)\Delta(x)^2$\\
    \raggedright
\end{itemize}
Ad un'attenta analisi possiamo vedere che tra i metodi di approssimazione in avanti, in indietro o centrale, il calcolo dell'errore portato dai primi due sono uguali tra di loro, quello centrale invece è diverso, esso dipende da un $\Delta(x)^2$ e non da un $\Delta(x)$, questo vuol dire che per $\Delta(x)<1$ funziona meglio, per $\Delta(x)>1$ funziona peggio. Nel caso in analisi $\Delta(x)=1$ quindi la scelta è indifferente.
\end{osservazione}
Per quanto riguarda la derivata seconda invece
\begin{osservazione}
Applicando la definizione di errore per la formula alle derivate finite per l'approssimazione della la derivata seconda:
$$
\epsilon= \frac{u_{i+1}-2u_{i}+u_{i-1}}{\Delta(x)^2} - u''_i
$$
Da questa rappresentazione non riusciamo purtroppo a dedurre nulla, proviamo quindi con un altro metodo. Consideriamo lo sviluppo di Taylor troncato al quarto ordine
$$
u_{i+1}=u_i+u'_i\Delta(x)+\frac{1}{2}u''_i\Delta(x)^2+\frac{1}{6}u'''_i\Delta(x)^3 + \frac{1}{24}u''''_i\Delta(x)^4 +o(\Delta(x)^4)
$$
$$
-\frac{1}{2}u''_i\Delta(x)^2=-u_{i+1} + u_i + u'_{i}\Delta(x) + \frac{1}{6}u'''_i\Delta(x)^3 + \frac{1}{24}u''''_i\Delta(x)^4 +o(\Delta(x)^4)
$$
$$
-\frac{1}{2}u''_i\Delta(x)^2 = -u_{i+1} + u_i + (\frac{u_{i+1}-u_{i-1}}{2\Delta(x)} -\frac{1}{6}u'''_i\Delta(x)^2)\Delta(x) + \frac{1}{6}u'''_i\Delta(x)^3 + \frac{1}{24}u''''_i\Delta(x)^4 +o(\Delta(x)^4)
$$
$$
-\frac{1}{2}u''_i\Delta(x)^2 = -u_{i+1} + u_i + \frac{1}{2}u_{i+1} - \frac{1}{2}u_{i-1} -\frac{1}{6}u''_i\Delta(x)^3 + \frac{1}{6}u'''_i\Delta(x)^3 + \frac{1}{24}u''''_i\Delta(x)^4 +o(\Delta(x)^4)
$$
$$
-\frac{1}{2}u''_i\Delta(x)^2 = -\frac{1}{2}u_{i+1} + u_i - \frac{1}{2}u_{i-1} + \frac{1}{24}u''''_i\Delta(x)^4 +o(\Delta(x)^4)
$$
$$
\epsilon=-u''_i + \frac{u_{i+1} - 2u_i + u_{i-1}}{\Delta(x)^2} \approx \frac{1}{12}u''''_i\Delta(x)^2
$$
Risulta quindi evidente che l'errore dipende da $\Delta(x)^2$. 
\end{osservazione}

%$$
%\frac{d^2u}{dx^2}\vline_{x_i}-\frac{u_{i+1}-u_i}{\Delta(x)}\approx \frac{\Delta(x)^2}{2} \frac{d^2u}{dx^2}\vline_\xi.
%$$

%Ma procediamo con ordine.\\

Per brevità di notazione poniamo d'ora in poi $\Delta(x)=h$.\\
\begin{proposizione}
Dato un problema differenziale $-u'' + \sigma u=f$ con condizioni al contorno di Dirichlet $g_0$ e $g_1$, allora se $\sigma\geq0$ la risoluzione del problema tramite metodo delle differenze finite è convergente.
\end{proposizione}

\begin{proof}
Senza ledere di generalità possiamo assumere che $u,f,\sigma$ definite su $I=(0,1)$, il problema assume quindi la forma

$$
\begin{cases}
-u'' + \sigma u=f\\
u(0)=g_0\\
u(1)=g_1
\end{cases}
$$

Per quanto visto nell'osservazione 2.1.3 possiamo scrivere
$$
u''(x_j) =\frac{u(x_{j+1}) - 2u(x_j) + u(x_{j-1})}{h^2} -\frac{1}{12} u^{(4)}(\xi_j)h^2
$$
dove $\xi_j$ e un punto opportuno in $(x_{j-1} , x_{j+1})$. Allora per $j = 1,...,N-1$, si può scrivere che

$$
\frac{-u(x_{j+1}) + 2u(x_j) -u(x_{j-1})}{h^2}
 + \frac{1}{12} u^{(4)}(\xi_j)h^2 + \sigma(x_j)u(x_j) = f(x_j) .
$$

Introducendo la notazione $\tau_{j}=\frac{1}{12} u''''(\xi_j)h^2$ (errore di troncamento locale) e ponendo:\\

$$\boldsymbol{u}=[u_x...u_{N-1}]^T$$
$$\boldsymbol{\tau_h}=[\tau_x...\tau_{N-1}]^T$$
$$\boldsymbol{b_h}=[(f(x_1) + \frac{g_0}{h^2},f(x_2),...,f(x_{N-2}),f(x_{N-1}) + \frac{g_1}{h^2}]^T$$

si può allora scrivere in forma matriciale,
$A_h\boldsymbol{u} = \boldsymbol{b_h} + \boldsymbol{\tau_h}$
dove $A_h = \frac{1}{h^2}
 tridiag(-1,2,-1) + diag(\sigma_1, ... ,\sigma_{N-1})$ e $\sigma_j = \sigma(x_j)$.\\
\vspace{1em}
Il metodo alle differenze finite consiste allora nel determinare un’approssimazione $u_h$ di
u andando a risolvere il sistema lineare
$A_h\boldsymbol{u_h} = \boldsymbol{b_h}.$
Osserviamo che $\boldsymbol{u_h}$ risulta ben definito in quanto $A_h$ è una matrice non singolare, il che è verificato in quando la matrice è composta da sole righe linearmente indipendenti.\\

\vspace{1em}

Risultando che $\boldsymbol{\tau_h}$ tende a zero quando h tende a zero, il metodo dicesi consistente. In particolare risulta $\boldsymbol{\tau_h} = O(h^2)$.\\
Tuttavia la consistenza non assicura da sola la convergenza del metodo.\\
Per studiarne la convergenza è necessario considerare il comportamento dell’errore $\boldsymbol{\epsilon_h} = \boldsymbol{u_h} -\boldsymbol{u}$ quando h tende a zero. Dato che risulta $A_h\boldsymbol{\epsilon_h} = \boldsymbol{\tau_h}$, e quindi $\boldsymbol{\epsilon_h} = A_h^{-1} \boldsymbol{\tau_h}$ si può quindi scrivere: $\left\|\boldsymbol{\epsilon_h}\right\|\geq \left\|A_h^{-1}\right\| \left\|\boldsymbol{\tau_h}\right\|$\\
Il passaggio successiovo è far vedere che, lavorando in norma infinito, si è in grado di trovare una costante che, per ogni $h$, maggiora $\left\|A_h^{-1}\right\|_{\infty} $. \\ 
A questo scopo osserviamo che si può dimostrare che sia $A_h$ che la matrice $A_{0h} = \frac{1}{h^2} tridiag(-1,2,-1)$ hanno inversa non negativa, e si ha: 

$$A_{0h}^{-1}-A_{h}^{-1}=A_{0h}^{-1}(A_h-A_{0h})A_h^{-1}\geq0.$$

Per come è definita $A_h$, vale la disuguaglianza $\left\|A_h^{-1}\right\|_{\infty} \leq \left\|A_{0h}^{-1}\right\|_{\infty} $
si può quindi osservare che
$\left\|A_{0h}^{-1}\right\|_{\infty} =max_j(A_{0h}^{-1}{ones(N-1)})_j$ dove ${ones(N-1)}$ indica il vettore di tutti 1.\\
Osservando che la soluzione esatta del problema $-u'' = 1, u(0) = u(1) = 0$, è il polinomio di secondo grado $\phi(x) = \frac{x(1-x)}{2}$, si può concludere che $A_{0h}^{-1}(ones(N-1)_j=\phi(x_j)$ e quindi che $\left\|A_h^{-1}\right\|_{\infty} \leq \left\|A_{0h}^{-1}\right\|_{\infty} \leq max_{0<x<1}|\phi_x|$.\\
Questo risultato di stabilità ci permette di concludere che l’errore $\boldsymbol{\epsilon_h}$ ha lo stesso ordine dell’errore di troncamento $\boldsymbol{\tau_h}$ e di conseguenza che il metodo è convergente del secondo ordine.\\
\end{proof}
\begin{osservazione}
Si noti che l’uniforme limitatezza della norma di $A_h^{-1}$ implica che il metodo numerico sia stabile.\\
\end{osservazione}
Questo è un risultato che ha valenza generale, riportato al caso dell'equazione del calore abbiamo che $\frac{\partial u}{\partial t} - \Delta u = \frac{\partial u}{\partial t} - \frac{\partial^2 u}{\partial x^2} - \frac{\partial^2 u}{\partial y^2} = 0$ per cui $-\frac{\partial^2 u}{\partial x^2}=\frac{\partial u}{\partial t}|_x$ e $-\frac{\partial^2 u}{\partial y^2}=\frac{\partial u}{\partial t}|_y$. Per entrambe le equazioni trovate ci troviamo nel caso $\sigma=0$ e quindi in particolare $\sigma\geq0$, in queste condizioni $\left\|A_h^{-1}\right\|_{\infty} = \left\|A_{0h}^{-1}\right\|_{\infty} $ e quindi in particolare $\left\|A_h^{-1}\right\|_{\infty} \leq \left\|A_{0h}^{-1}\right\|_{\infty} $, la condizione di convergenza è dunque soddisfatta.


\newpage

\subsection{L'equazione del calore}

\raggedright

Il metodo delle differenze finite sarà impiegato in uno script MATLAB per l'implementazione dell'equazione del calore, è bene quindi prenderla in esame.


$$
\begin{cases}
\frac{\partial u}{\partial t}(t,x)-\Delta u(t,x) = 0 \ x \in \mathbb R^2, t\ge 0 \ .\\ 
u(0,x) = u_0(x)\ . \\
\end{cases}
$$

Ricordiamo $\Delta u=\frac{\partial^2u}{\partial x^2}+\frac{\partial^2u}{\partial y^2}$ per cui la sua approssimazione numerica sarà:
$$
u_{i,j}^{n+1}=u_{i,j}^{n}+k_x \Delta t(u_{i+1,j}^{n}-2u_{i,j}^{n}+u_{i-1,j}^{n})+k_y \Delta t(u_{i,j+1}^{n}-2u_{i,j}^{n}+u_{i,j-1}^{n})
$$
Con $k_x$ e $k_y$ coefficienti di controllo (nello script porremo $k_x=k_y=k$).\\
L'equazione del calore, come anticipato, determina l'evoluzione di un sistema isolato che presenta al suo interno una data distribuzione di calore. \`E banale pensare che con il passare del tempo il calore si distribuisca, tendendo per un tempo infinito ad una distribuzione uniforme.

\vspace{1em}

Applicando l'equazione del calore si ottiene quindi un'immagine sempre più \textit{"liscia"}, di fatto una sfocatura, e per un numero di iterazioni idealmente infinito si giungerebbe ad una distribuzione uniforme di colore, ossia una tinta unita.

\vspace{1em}
%\newpage

Il seguente script MATLAB illustra come questo processo opera nella pratica.

\begin{lstlisting}[language=MATLAB]
Im=imread('parrot.jpeg');   %Apro l'immmagine
Im=rgb2gray(Im);            %La trasformo in bianco e nero
Im=imnoise(Im,'gaussian');  %Aggiungo del rumore

%---Definisco le costanti e le condizioni iniziali

[ny, nx, ~]=size(Im)        %Dimensioni dell'immagine
dt=0.25;                    %Passo temporale
u=double(Im);               %Copia dell'immagine originale su cui                                lavorare
T=3			                    %Tempo, ossia T/dt + 1 definisce il numero                           di iterazioni da eseguire
k=0.5;

%---Mostro l'immagine originale
imshow(uint8(Im))
title('immagine originale'); 

%---Metodo eq del calore
u=double(Im);
for t = 0:dt:T
   u_xx = u(:,[2:nx nx],:) - 2*u + u(:,[1 1:nx-1],:);  % derivata                                                           seconda lungo x
   u_yy = u([2:ny ny],:,:) - 2*u + u([1 1:ny-1],:,:);  % derivata                                                           seconda lungo y
   u= u + k*dt*(u_xx+u_yy);
   temp=u;
end

\end{lstlisting}
%\vspace{1em}
\newpage
Provando a cambiare il tempo, ossia il numero di iterazioni, si può osservare come un maggior lasso di tempo produca immagini più sfocate.

\begin{figure} 
\centering
\includegraphics[scale=0.255, trim = 8.9cm 16.9cm 6.9cm 14.9cm, clip]{Pictures/Risultati/eq del calore_striscia.png}
%\includegraphics[scale=0.27, trim = 1.9cm 0.0cm 1.9cm 0.6cm, clip]{Pictures/Risultati/eq del calore 1.png}
%\includegraphics[scale=0.27, trim = 1.9cm 1.3cm 1.9cm 10.3cm, clip]{Pictures/Risultati/eq del calore.png}
\caption{Effetti dell'equazione del calore nel tempo.}\label{fig:figura}
\end{figure}


Questo metodo però ha un importante difetto, il rumore viene effettivamente eliminato o almeno ridotto ma si perdono importanti dettagli. In particolare, i margini degli oggetti presenti nell'immagine tendono progressivamente ad attenuarsi, come diretta conseguenza delle proprietà di regolarizzazione della soluzione dell'equazione del calore. Esistono tuttavia procedimenti come il metodo Perona-Malik che sono decisamente più efficaci.\\


\subsection{Studio della stabilità del metodo}
La tecnica delle differenze finite fornisce uno schema numerico che non è incondizionatamente stabile. Per studiare la sua stabilità e, conseguentemente, la sua convergenza, (grazie al teorema di equivalenza di Lax) utilizziamo un metodo classico per lo studio delle approssimazioni per le PDE:  il metodo di Von Neumann.\footnote{\cite{stab}}\\
Il metodo di Von Neuman si basa sulla decomposizione degli errori in serie di Fourier\\
Prima di procedere con il calcolo è utile fare un'osservazione preliminare.
\begin{osservazione}
Noto che il problema è risolubile tramite la formula iterativa
$$
u_{i,j}^{n+1}=u_{i,j}^{n}+r_{x}(u_{i+1,j}^{n}-2u_{i,j}^{n}+u_{i-1,j}^{n})+r_{y}(u_{i,j+1}^{n}-2u_{i,j}^{n}+u_{i,j-1}^{n})
$$
essa può essere riscritta come
$$
u_{i,j}^{n+1}=\frac{1}{2}u_{i,j}^{n}+r_{x}(u_{i+1,j}^{n}-2u_{i,j}^{n}+u_{i-1,j}^{n})+\frac{1}{2}u_{i,j}r_{y}(u_{i,j+1}^{n}-2u_{i,j}^{n}+u_{i,j-1}^{n})
$$
risultando essere somma di due fattori speculari riguardanti le due coordinate, possiamo quindi ragionevolmente effettuare i calcoli per una delle due parti e i risultati saranno coerenti anche nell'altra parte.\\
Detto $\epsilon=N-u$ l'errore dove u è la soluzione calcolata dall'algoritmo in precisione infinita e N è la soluzione calcolata in precisione finita di calcolo, allora
\hspace{0.5em}
$u_{j}^{n+1}=\frac{1}{2}u_{j}^{n}+r(u_{j+1}^{n}-2u_{j}^{n}+u_{j-1}^{n})$
e
$N_{j}^{n+1}=\frac{1}{2}N_{j}^{n}+r(N_{j+1}^{n}-2N_{j}^{n}+N_{j-1}^{n})$
da cui 
\hspace{-0.25em}
$$
\epsilon_{j}^{n+1}=N_{j}^{n+1}-u_{j}^{n+1}=\frac{1}{2}N_{j}-\frac{1}{2}u_{j}^{n}+r(N_{j+1}-u_{j+1}^{n}-2N_{j}+2u_{j}^{n}+N_{j-1}-u_{j-1}^{n})=\frac{1}{2}\epsilon _{j}^{n}+r(\epsilon_{j+1}^{n}-2\epsilon _{j}^{n}+\epsilon _{j-1}^{n})
$$
Sommando le due parti ritroviamo
$$
\epsilon_{i,j}^{n+1}=\epsilon_{i,j}^{n}+r_{x}(\epsilon_{i+1,j}^{n}-2\epsilon_{i,j}^{n}+\epsilon_{i-1,j}^{n})+r_{y}(\epsilon_{i,j+1}^{n}-2\epsilon_{i,j}^{n}+\epsilon_{i,j-1}^{n})
$$
questo dimostra che la soluzione e l'errore hanno lo stesso andamento.\\
\end{osservazione}
Questa osservazione, oltre a fornirci un risultato necessario per il calcolo che ci accingiamo a fare, ci introduce al ragionamento per \textit{separazione delle variabili} che utilizzeremo nella dimostrazione della proposizione seguente ed è molto utile per semplificare i calcoli.\\
\begin{proposizione}
il problema
$$
u_{i,j}^{n+1}=u_{i,j}^{n}+r_{x}(u_{i+1,j}^{n}-2u_{i,j}^{n}+u_{i-1,j}^{n})+r_{y}(u_{i,j+1}^{n}-2u_{i,j}^{n}+u_{i,j-1}^{n})
$$
dove $r_x=k_x\frac{\Delta t}{(\Delta x)^2}$ e $r_y=k_y\frac{\Delta t}{(\Delta y)^2}$, risulta stabile $\Leftrightarrow r_x+r_y\leq\frac{1}{2}$ 
\end{proposizione}
\begin{proof}

Senza ledere di generalità, consideriamo il caso dell'equazione del calore \textbf{uno-dimensionale}, come visto essa può essere discretizzata come:
$$
u_{j}^{n+1}=u_{j}^{n}+r(u_{j+1}^{n}-2u_{j}^{n}+u_{j-1}^{n})
$$
dove $r=k \frac{\Delta t}{(\Delta x)^2}$.\\


%Per le equazioni differenziali lineari con condizioni al contorno periodiche, la variazione spaziale dell'errore può essere espansa in una serie finita di Fourier rispetto a x nell'intervallo L, cioè
%$$
%\epsilon (x,t)=\sum _{m=-M}^{M}E_{m}(t)e^{{i}k_{m}x}
%$$
%dove $k_{m}={\frac {\pi m}{L}}$ è il numero d'onda e $M=\frac{L}{\Delta x}$

La variazione spaziale dell'errore può essere espansa con un integrale di Fourier rispetto ad x, cioè:
$$
\epsilon (x,t)=\int _{\tilde{\nu}=-{\frac {\pi }{\Delta x}}}^{\tilde{\nu}={\frac {\pi }{\Delta x}}}E_{\tilde{\nu}}(t)e^{{i}\tilde{\nu}x}d\tilde{\nu}
$$
Essendo l'equazione lineare, il termine generico va come l'integrale stesso, ossia: 
$\epsilon _{j}^{n}=E_{\tilde{\nu}}(t)e^{i\tilde{\nu}_{m}x}$ da cui:\\
\vspace{0.5em}
$\epsilon _{j}^{n+1}=E_{\tilde{\nu}}(t+\Delta t)e^{i\tilde{\nu}x}$\\ 
$\epsilon _{j+1}^{n}=E_{\tilde{\nu}}(t)e^{i\tilde{\nu}(x+\Delta x)}$\\ 
$\epsilon _{j-1}^{n}=E_{\tilde{\nu}}(t)e^{i\tilde{\nu}(x-\Delta x)}$\\
Sostituindo questi valori in $\epsilon_{j}^{n+1}=\epsilon _{j}^{n}+r(\epsilon _{j+1}^{n}-2\epsilon _{j}^{n}+\epsilon _{j-1}^{n})$ si ottiene:

$$
E_{\tilde{\nu}}(t+\Delta t)e^{i\tilde{\nu}x}=E_{\tilde{\nu}}(t)e^{i\tilde{\nu}x}+r(E_{\tilde{\nu}}(t)e^{i\tilde{\nu}(x+\Delta x)}-2E_{\tilde{\nu}}(t)e^{i\tilde{\nu}}x+E_{\tilde{\nu}}(t)e^{i\tilde{\nu}(x-\Delta x)})
$$
Affinché il metodo sia stabile occorre che l'errore non aumenti mai, ossia che\\
\vspace{0.25em}
$|E_{\tilde{\nu}}(t+\Delta t)|\leq |E_{\tilde{\nu}}(t)| \Rightarrow |\frac{E_{\tilde{\nu}}(t+\Delta t)}{E_{\tilde{\nu}}(t)}|\leq 1$. Esplicitiamo quindi per $\frac{E_{\tilde{\nu}}(t+\Delta t)}{E_{\tilde{\nu}}(t)}$ ed otteniamo:

$$
\frac {E_{\tilde{\nu}}(t+\Delta t)}{E_{\tilde{\nu}}(t)}=1+r(e^{i\tilde{\nu}\Delta x}+e^{-i\tilde{\nu}\Delta x}-2).
$$
detto $\theta =\tilde{\nu}\Delta x$, ricordo $\tilde{\nu}\in [-\frac{\pi}{\Delta x},\frac{\pi}{\Delta x}] \Rightarrow \theta \in [-\pi ,\pi ]$
per cui l'equazione diventa
$$
\frac {E_{\tilde{\nu}}(t+\Delta t)}{E_{m}(t)}=1+r(e^{i\theta x}+e^{-i\theta}-2).
$$

Prendiamo ora in considerazione l'identità 
$$
\sin \left({\frac {\theta }{2}}\right)={\frac {e^{i\theta /2}-e^{-i\theta /2}}{2i}} \Rightarrow
\sin ^{2}\left({\frac {\theta }{2}}\right)=-{\frac {e^{i\theta }+e^{-i\theta }-2}{4}}
$$
alla luce di questa osservazione, l'equazione diventa:
$$
\frac {E_{\tilde{\nu}}(t+\Delta t)}{E_{\tilde{\nu}}(t)}=1-4r\sin^2(\frac{\theta}{2}).
$$

Come detto, il metodo è stabile $\Leftrightarrow|\frac{E_{\tilde{\nu}}(t+\Delta t)}{E_{\tilde{\nu}}(t)}|\leq 1$ ma visto che $\frac {E_{\tilde{\nu}}(t+\Delta t)}{E_{\tilde{\nu}}(t)}=1-4r\sin^2(\frac{\theta}{2})\Rightarrow$ il metodo risulta stabile $\Leftrightarrow|1-4r\sin^2(\frac{\theta}{2})|\leq1$. Esplicitando:
$$
-1\leq 1-4r\sin^2(\frac{\theta}{2})\leq 1 \Rightarrow-2\leq -4r\sin^2(\frac{\theta}{2})\leq 0 \Rightarrow 0\leq 4r\sin^2(\frac{\theta}{2})\leq 2
$$
ma $\sin^2(\frac{\theta}{2})\ge0$ $\forall\theta\Rightarrow$il metodo risulta stabile $\Leftrightarrow r\sin^2(\frac{\theta}{2})\leq\frac{1}{2}$ siccome $\sin^2(\frac{\theta}{2})\leq1$ $\forall\theta\Rightarrow$ la condizione è sicuramente soddisfatta se $r\leq\frac{1}{2}$.
Avendo definito $r=k \frac{\Delta t}{(\Delta x)^2} \Rightarrow$ condizione sufficiente affinché il metodo sia stabile è $k \frac{\Delta t}{(\Delta x)^2}\leq\frac{1}{2}$\\
\vspace{1em}
Nel caso 2-dimensionale la soluzione discreta della PDE è:
$$
u_{i,j}^{n+1}=u_{i,j}^{n}+r_{x}(u_{i+1,j}^{n}-2u_{i,j}^{n}+u_{i-1,j}^{n})+r_{y}(u_{i,j+1}^{n}-2u_{i,j}^{n}+u_{i,j-1}^{n})
$$
dove $r_{x}=k_x \frac{\Delta t}{(\Delta x)^2}$ e $r_{y}=k_y \frac{\Delta t}{(\Delta y)^2}$\\ per cui la condizione diventa $r_{x}+r_{y}=k_x \frac{\Delta t}{(\Delta x)^2}+k_y \frac{\Delta t}{(\Delta y)^2}\leq\frac{1}{2}$.\\

\end{proof}
Nel caso in esame $\Delta x=\Delta y=1$ e $ k_x=k_y \Rightarrow r_{x}+r_{y}=k \Delta t+k \Delta t\leq\frac{1}{2} \Rightarrow k\Delta t\leq \frac{1}{4}$.\\
\vspace{0.5em}
La condizione appena calcolata è detta \textbf{condizione di Courant-Friedrichs-Lewy} (abbreviato \textbf{CFL}) per le equazioni alle derivate parziali di tipo parabolico




%\bibliography{references}



\end{document}

\







