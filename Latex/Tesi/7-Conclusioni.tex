\newpage
\section{Conclusioni}
La trattazione svolta ha permesso di utilizzare alcuni risultati di diverse branche della matematica per approcciarsi ad un problema reale. Si sono visti i concetti basilari che permettono l'elaborazione digitale delle immagini tramite un calcolatore quali i concetti di risoluzione, pixel e spazio di memoria occupato dalle immagini al variare di risoluzione e numero dei colori. Si sono poi introdotti i concetti di convoluzione e soluzione fondamentale di un operatore differenziale, strumenti imprescindibili nello studio delle equazioni alle derivate parziali. Sono stati introdotti alcuni semplici problemi di filtraggio digitale risolubili tramite strumenti di algebra lineare. Sono stati altresì approfonditi i metodi numerici, come il metodo delle differenze finite con relativo errore di troncamento e il metodo di Eulero esplicito, necessari all'implementazione del metodo Perona-Malik, ricavandone anche le condizioni di stabilità. Lungo la trattazione sono stati inoltre richiamati i concetti di analisi matematica utilizzati, come derivate, gradiente e laplaciano. Grazie a questi concetti matematici è stato possibile, sfruttando anche strumenti tipici dell'elaborazione digitale delle immagini come le maschere di convoluzione, ottenere risultati semplici come la diffusione di un'immagine tramite equazione del calore e rilevamento dei bordi, che hanno fatto da base per l'implementazione del filtro di diffusione anisotropica in esame.
In fine sono stati mostrati diversi esempi applicativi che mostrano l'operato dello script MATLAB ottenuto.\\
Lo studio di questo metodo ha storicamente aperto le porte a successivi studi in questo campo, portando allo sviluppo di numerosi modelli di diffusione non lineare. Di particolare rilievo è il metodo proposto da Cottet e Germain nel 1993\footnote{\cite{Weickert}}, che si differenzia dal metodo Perona-Malik adottando una diffusione regolata da un sistema di tensori portando ad una diffusione diversa lungo le diverse direzioni definite dalla geometria locale dell'immagine, ottenendo quindi un metodo di diffusione anisotropica in senso proprio. In questo approccio, l'introduzione di una scala di integrazione nel tensore di struttura è una caratteristica essenziale del modello. Successivamente (nel 1996) Cottet e El Ayyadi hanno proposto un modello di restauro dell'immagine modificato che sostituisce la regolarizzazione spaziale con una regolazione temporale.\\
Un altro tipo di schema di filtraggio anisotropico è stato proposto da Yang\footnote{\cite{YANG}} e i suoi collaboratori. Invece di utilizzare i gradienti locali come mezzo per controllare l'anisotropia del filtro, con il metodo di Yang si costruisce il kernel del filtro in base alle caratteristiche anisotropiche locali dell'immagine e il loro schema di stima dell'orientamento si basa sul fatto che lo spettro di potenza di un pattern orientato giace lungo una linea che passa per l'origine di un dominio di Fourier.
%la diffusione è controllata dall'orientamento locale dell'intensità e da una misura anisotropa.
Tuttavia le stime dei parametri nello schema di filtraggio di Yang sono
non sono ottimali per le caratteristiche delle immagini, come angoli, giunzioni o bordi.