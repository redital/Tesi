Le immagini sono sempre state fondamentali per l'uomo, che da sempre le utilizza per ricordare, illustrare, comunicare, etc. Per questo motivo, sfruttarle nel migliore dei modi è sempre stato un problema di grande interesse. Esistono dei metodi di elaborazione digitali detti "di filtraggio" il cui scopo è modificare l'immagine, estraendone alcuni elementi, nascondendoli o facendoli risaltare così da migliorarla, cioè renderla quanto più utile agli scopi richiesti.\\
Durante questa trattazione, dopo una breve intoduzione che servirà a dare una visione d'insieme del campo nel quale ci si sta introducendo, sarà illustrato il concetto di immagine digitale, di filtro ed in particolare di equazioni alle derivate parziali, di come esse giochino un ruolo fondamentale nella scrittura dei filtri e la loro implementazione al calcolatore.\\
Nello specifico, nella parte introduttiva si parlerà di storia delle immagini digitali e del filtraggio analogico e digitale, dei principali problemi affrontati e di alcune delle più semplici elaborazioni digitali. Verranno fornite alcune nozioni introduttive, come il concetto di immagine come funzione matematica e quindi di immagine digitale, capire come è codificata, si parlerà quindi di pixel, del concetto di risoluzione e di quanto essa influisca sullo spazio di archiviazione. In fine saranno introdotti i filtri, intesi come convoluzioni tra la funzione immagine e la funzione filtro.\\
Inizierà poi la trattazione più specifica, che sarà il fulcro di questo studio, ossia l'implementazione del metodo Perona-Malik. Per fare ciò verrà illustrato il metodo delle differenze finite per l'approssimazione delle derivate con relativa analisi dell'errore di troncamento. Questo risultato sarà sfruttato per l'implementazione dell'equazione del calore tramite uno script MATLAB che sarà poi modificato per implementare il metodo Perona-Malik.   