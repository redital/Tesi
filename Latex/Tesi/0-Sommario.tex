Le immagini sono da sempre state fondamentali per l'uomo, che sin dall'antichità le ha utilizzate per ricordare, illustrare, comunicare, etc. In tempi più recenti, questo fenomeno ha acquistato progressivamente centralità nella società contemporanea, spesso definita 'socità dell'immagine'. La manipolazione, l'archiviazione e il trasferimento di contenuto attraverso le immagini, è una attività costantemente praticata e, per questo motivo, lo sviluppo di tecniche sofisticate per tali scopi è un problema di grande interesse in vari campi della scienza e della tecnica. In questo lavoro ci occuperemo principalmente di tecniche digitali di filtraggio. Lo scopo di tali tecniche è modificare l'immagine, estraendone elementi, nascondendoli o facendoli risaltare, così da migliorarla per renderla quanto più utile possibile agli scopi richiesti.\\
Durante questa trattazione, dopo una breve introduzione volta a fornire una visione d'insieme del campo in esame, verrà illustrato il concetto di immagine digitale, di filtro e di equazioni alle derivate parziali, con particolare riguardo al ruolo fondamentale che queste ultime giocano nella scrittura dei filtri e la relativa implementazione all'interno del calcolatore.\\
Nello specifico, nella parte introduttiva si parlerà della storia delle immagini digitali e del filtraggio analogico e digitale, dei principali problemi consequenziali e di alcune delle più semplici elaborazioni digitali. Verranno poi fornite alcune nozioni preliminari tra cui: il concetto di immagine come funzione matematica e quindi di immagine digitale, della relativa codifica, di pixel, del concetto di risoluzione ed altresì di quanto essa influisca sullo spazio di archiviazione. Successivamente saranno analizzati i filtri intesi come convoluzioni tra la funzione immagine e la funzione filtro.\\
Infine la trattazione diventerà più specifica, presentando il fulcro di questo studio, ossia la presentazione di un metodo di filtraggio anisotropico basato sulle equazioni alle derivate parziali comunemente chiamato metodo Perona Malik. A sostegno di ciò verrà illustrato il metodo delle differenze finite per l'approssimazione delle derivate parziali con relativa analisi dell'errore di troncamento. Questo risultato sarà dunque sfruttato per la realizzazione di uno script MATLAB che operi l'equazione del calore e che verrà poi opportunamente modificato per implementare il metodo Perona-Malik.   