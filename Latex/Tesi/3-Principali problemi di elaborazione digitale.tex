\chapter{Principali problemi di elaborazione digitale}

%\section{Principali problemi di elaborazione digitale}

Al giorno d'oggi i computer permettono una vasta gamma di elaborazioni digitali, ce ne sono alcuni, più semplici ma di notevole interesse, ormai già perfezionati ed altri che sono ancora oggetto di studio.

\section{Rotazioni, riflessioni,etc}
Le trasformazioni più semplici in assoluto riguardano movimenti rigidi come la \textbf{traslazione} o la \textbf{rotazione}, o anche omeomorfismi come la \textbf{riflessione}. Questi sono molto utili per introdurre i primi concetti matematici, come l'impiego di matrici.\\
Definiamo una \textbf{matrice di trasformazione} che moltiplicata per un vettore di coordinate ci restituisca un altro vettore di coordinate. Questo vuol dire che quel punto va spostato dalle coordinate in cui si trovava a quelle appena calcolate. Ad esempio:

$$
{\begin{bmatrix}
-1&0&0\\
0&1&0\\
0&0&1
\end{bmatrix}}
$$

\noindent
\`E una matrice di riflessione lungo l'asse verticale, infatti: 

$$
{\begin{bmatrix}
-1&0&0\\
0&1&0\\
0&0&1
\end{bmatrix}}
\begin{bmatrix}
x\\y\\1
\end{bmatrix}
=
\begin{bmatrix}
-x&y&1
\end{bmatrix}
$$	

\vspace{1em} \noindent
Cioè ogni punto rimane alla stessa quota ma cambia la propria x con -x, che è esattamente ciò che si intende per riflessione lungo l'asse verticale.

\newpage \noindent 
Altri esempi possono essere:

\begin{align*}
&\begin{bmatrix}
1&0&0\\
0&-1&0\\
0&0&1
\end{bmatrix}&~&
\text{Riflessione lungo l'asse orizzontale}\\
\\
&\begin{bmatrix}
2&0&0\\
0&1.5&0\\
0&0&1
\end{bmatrix}&~&
\text{Scalamento}\\
\\
&\begin{bmatrix}
\cos(\theta )&\sin(\theta )&0\\
-\sin(\theta )&\cos(\theta )&0\\
0&0&1
\end{bmatrix}&~&
\text{Rotazione di un angolo theta}\\
\end{align*}



\section{Cambio prospettiva}
Una delle trasformazioni più semplici è quella del cambio prospettiva. Una volta compresa questa trasformazione si è fatto anche un primo passo per parlare di warping, che sarà trattato più avanti.

\vspace{1em} \noindent
Un esempio pratico può essere: si ha la foto di un documento poggiato su una scrivania e la si vuole migliorare in modo da estrarne solo il documento e che i suoi bordi concidano quindi con i bordi dell'immagine.

\vspace{1em} \noindent
In genere la cosa più semplice ed affidabile è quella di far scegliere i 4 angoli del documento ad un utente. Volendo automatizzare il processo si può però scegliere un'altra strada, ossia estrarre i bordi dell'immagine e cercare tra questi quelli che formano un trapezio; presumibilmente quello sarà il documento.

\vspace{1em} \noindent
Ottenuti i 4 angoli, occorre calcolare la lunghezza e la larghezza della nuova immagine. Per fare ciò si può calcolare la lunghezza dei 4 lati banalmente con il teorema di Pitagora. A questo punto, presi i lati a due a due non adiacenti, cioè che non hanno vertici in comune, ne si confrontano le lunghezze.\\
Per ogni coppia si adotta la lunghezza maggiore (si può anche usare quella minore ma è una scelta di scarso interesse per lo studio in atto).

\vspace{30em} \noindent
Ottenuti questi dati si può calcolare la matrice di trasformazione. Essa sarà del tipo:

\begin{align*}
&\begin{bmatrix}
a_1&a_2&b_1\\
a_3&a_4&b_2\\
c_1&c_2&1
\end{bmatrix}
\text{ dove }
\begin{bmatrix}
a_1&a_2\\
a_3&a_4
\end{bmatrix}
\text{ definisce rotazione e scalamento, } 
\begin{bmatrix}
b_1\\
b_2
\end{bmatrix}
\text{ è un vettore di}\\
&\text{traslazione e} 
\newline
\begin{bmatrix}
c_1&c_2
\end{bmatrix}
\text{ è un vettore di proiezione (che è nullo se il riquadro iniziale e quello}\\
&\text{finale coincidono).}\\
\end{align*}

\vspace{-3em}

\begin{align*}
&\text{L'intento è che, moltiplicando tale matrice per il vettore  }
\begin{bmatrix}
\text{coordinata x}\\
\text{coordinata y}\\
1
\end{bmatrix}
\text{, si ottiene il vettore}\\
&\begin{bmatrix}
\text{nuova coordinata x*k}\\
\text{nuova coordinata y*k}\\
k
\end{bmatrix}
\text{ eseguendo questa operazione per tutti i punti, Si costruisce }\\
&\text{l'immagine desiderata.}
\end{align*}

\section{Morphing}
Il morphing consiste nella trasformazione fluida, graduale tra due immagini di forma diversa, raffiguranti oggetti, persone, volti, paesaggi.

\vspace{1em} \noindent
Il morphing è quindi una tecnica che combia l'uso in contemporanea di una dissolvenza incrociata e di un effetto di deformazione chiamato \textbf{warping} (termine inglese che significa appunto deformazione).

\vspace{1em} \noindent
Per operare il warping si definiscono sull'immagine di partenza dei "punti chiave" che possono essere uniti tra di loro con delle linee e si definiscono sull'immagine di destinazione i corrispondenti punti e di conseguenza le corrispondenti linee.\\ 
\begin{figure}[htb] \centering
\includegraphics[scale=0.5, trim = 0 1.1cm 0 0, clip]{Pictures/Striscia_morphing.jpg}
\caption{Processo di morphing con alcuni risultati intermedi.}\label{fig:figura}
\end{figure}

\noindent
Durante la dissolvenza dall'immagine iniziale a quella finale, le immagini vengono deformate facendo in modo che ciascun punto chiave si muova lungo il percorso che porta dalla sua posizione nell'immagine di partenza alla posizione del corrispondente punto nell'immagine di arrivo.